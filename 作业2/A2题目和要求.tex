\documentclass[a4paper, 11pt]{article}
\usepackage{comment} % enables the use of multi-line comments (\ifx \fi) 
\usepackage{lipsum} %This package just generates Lorem Ipsum filler text. 
\usepackage{fullpage} % changes the margin
\usepackage[a4paper, total={7in, 10in}]{geometry}
\usepackage[fleqn]{amsmath}
\usepackage{amssymb,amsthm}  % assumes amsmath package installed
\newtheorem{theorem}{Theorem}
\newtheorem{corollary}{Corollary}
\usepackage{graphicx}
\usepackage{caption}
\usepackage{bm}

% \usepackage[pdftex]{graphicx}
\usepackage{tikz}
\usetikzlibrary{arrows}
\usepackage{verbatim}
\usepackage[numbered]{mcode}
\usepackage{float}
\usepackage{tikz}
    \usetikzlibrary{shapes,arrows}
    \usetikzlibrary{arrows,calc,positioning}

    \tikzset{
        block/.style = {draw, rectangle,
            minimum height=1cm,
            minimum width=1.5cm},
        input/.style = {coordinate,node distance=1cm},
        output/.style = {coordinate,node distance=4cm},
        arrow/.style={draw, -latex,node distance=2cm},
        pinstyle/.style = {pin edge={latex-, black,node distance=2cm}},
        sum/.style = {draw, circle, node distance=1cm},
    }
\usepackage{xcolor}
\usepackage{mdframed}
\usepackage[shortlabels]{enumitem}
\usepackage{indentfirst}
\usepackage{hyperref}
\usepackage{CJKutf8}
    
\renewcommand{\thesubsection}{\thesection.\alph{subsection}}

\newenvironment{problem}[2][Q]
    { \begin{mdframed}[backgroundcolor=gray!20] \textbf{#1 #2} \\}
    {  \end{mdframed}}

% Define solution environment
\newenvironment{solution}
    {\textit{Solution:}}
    {}

\renewcommand{\qed}{\quad\qedsymbol}
%%%%%%%%%%%%%%%%%%%%%%%%%%%%%%%%%%%%%%%%%%%%%%%%%%%%%%%%%%%%%%%%%%%%%%%%%%%%%%%%%%%%%%%%%%%%%%%%%%%%%%%%%%%%%%%%%%%%%%%%%%%%%%%%%%%%%%%%
\begin{document}
\begin{CJK}{UTF8}{gbsn}
%Header-Make sure you update this information!!!!
\noindent
%%%%%%%%%%%%%%%%%%%%%%%%%%%%%%%%%%%%%%%%%%%%%%%%%%%%%%%%%%%%%%%%%%%%%%%%%%%%%%%%%%%%%%%%%%%%%%%%%%%%%%%%%%%%%%%%%%%%%%%%%%%%%%%%%%%%%%%%
\large\textbf{课程: 人工智能} \hfill \textbf{作业 5}   \\
北航软件学院 \\
\normalsize 学期: 2025,春季\hfill 提交截止时间:  2025年5月9日, 11:59 PM \\
\noindent\rule{7in}{2.8pt}
\textbf{提醒注意:}
\begin{itemize}
\item 本次作业截止于2025年5月9日。
\item 作业分为三部分:问答题、实训题、以及实训题报告
\begin{itemize}
    \item 问答题答案可以手写并扫描,或者用latex(或word)手打,最终以QA.pdf文件命名。
    \item 实训题按照项目共享链接内要求和基础代码进行作答。
    \item 报告部分同样可以手写或者手打,以Report.pdf文件命名。
    \item 作业提交格式:$<student ID>$\_$<name>$\_A2.zip。比如1921102\_田嘉怡\_A2.zip
    \item 提交的zip文件要求(仅)包括:
    \begin{itemize}
        \item 实训题文件:包括 main.ipynb,以及predict.csv文件。
        \item 问答题答案:QA.pdf
        \item 报告:Report.pdf
    \end{itemize}
    
\end{itemize}
\item 作业压缩包需要在spoc平台上提交。
\item 每迟交1天(不满1天按1天计算),本次作业扣除10\%分数。
\item 不按作业要求和格式提交,视情况扣分。不得抄袭。
\end{itemize}

\noindent\rule{7in}{1pt}
\textbf{第一部分:问答题}(4分)

%%%%%%%%%%%%%%%%%%%%%%%%%%%%%%%%%%%%%%%%%%%%%%%%%%%%%%%%%%%%%%%%%%%%%%%%%%%%%%%%%%%%%%%%%%%%%%%%%%%%%%%%%%%%%%%%%%%%%%%%%%%%%%%%%%%%%%%%
% Problem 1
%%%%%%%%%%%%%%%%%%%%%%%%%%%%%%%%%%%%%%%%%%%%%%%%%%%%%%%%%%%%%%%%%%%%%%%%%%%%%%%%%%%%%%%%%%%%%%%%%%%%%%%%%%%%%%%%%%%%%%%%%%%%%%%%%%%%%%%%


\begin{problem}{1 神经网络基础 (1分)}
围绕神经网络基础,回答以下问题:
\begin{enumerate}
\item 给定一个3层全连接网络(输入层2节点,隐藏层3节点,输出层1节点),写出前向传播的矩阵运算形式(使用Sigmoid激活)。
\item 交叉熵损失函数如何衡量预测概率分布与真实分布的差异?写出二分类问题的交叉熵公式。
\end{enumerate}
\end{problem}


%%%%%%%%%%%%%%%%%%%%%%%%%%%%%%%%%%%%%%%%%%%%%%%%%%%%%%%%%%%%%%%%%%%%%%%%%
% Problem 2
%%%%%%%%%%%%%%%%%%%%%%%%%%%%%%%%%%%%%%%%%%%%%%%%%%%%%%%%%%%%%%%%%%%%%%%%%%%%%%%%%%%%%%%%%%%%%%%%%%%%%%%%%%%%%%%%%%%%%%%%%%%%%%%%%%%%%%%%
\begin{problem}{2 模型优化 (1分)}
围绕深度学习模型优化,回答以下问题:
\begin{enumerate}
\item 在前馈神经网络中,所有的参数能否被初始化为0?如果不能,能否全部初始化为其他相同的值?原因是什么?
\item 给定损失函数$L(w)=w^2+2w+1$,计算权重$w$在初始值$w_0=3$时的梯度,并说明梯度下降的更新方向。
\end{enumerate}
\end{problem}


%%%%%%%%%%%%%%%%%%%%%%%%%%%%%%%%%%%%%%%%%%%%%%%%%%%%%%%%%%%%%%%%%%%%%%%%%
% Problem 3
%%%%%%%%%%%%%%%%%%%%%%%%%%%%%%%%%%%%%%%%%%%%%%%%%%%%%%%%%%%%%%%%%%%%%%%%%%%%%%%%%%%%%%%%%%%%%%%%%%%%%%%%%%%%%%%%%%%%%%%%%%%%%%%%%%%%%%%%


\begin{problem}{3 激活函数 (2分)}
logistic sigmoid函数在深度学习中用途很广。根据函数表达式回答下列问题:
\begin{equation}
    \sigma(x)=\frac{1}{1+e^{-x}}
\end{equation}
\begin{enumerate}
\item 证明$1-\sigma(x)=\sigma(-x)$;\\
\item 证明 $\sigma'(x)=\sigma(x)(1-\sigma(x))$ ,其中$\sigma'(x)=\frac{d}{dx}\sigma(x)$。并画出$\sigma(x)$和$\sigma'(x)$的函数图像;\\
\item 神经网络通常都是layer-by-layer处理,比如,对于输入$\bm{x}$,输出$\bm{y}$可以用下列式子表示:
\begin{equation}
    \bm{y}=f^{(L)}(f^{(L-1)}(f^{2}(f^{1}(x))))
\end{equation}
式子中$f^{(i)}$代表着神经网络中的第$i$层,当神经网络层数(式子中的L)很大的时候,我们称为深度神经网络。
神经网络优化算法一般选择随机梯度下降算法,我们用$\bm{\theta}$表示神经网络中的参数,$\bm{g}$表示神经网络对应损失函数的梯度,神经网络参数更新用下式表示:
\begin{equation}
    \bm{\theta}^{new}\leftarrow \bm{\theta}-\lambda \bm{g}
\end{equation}
式子中$\lambda$是学习率。梯度计算采用链式法则。设$\bm{\theta}^{(i)}$为第i层的参数,$\bm{y}^{(i)}$是i层的输出:
\begin{equation}
    \frac{\partial l}{\partial(\bm{\theta}^{(i)})^{T}}
    =\frac{\partial l}{\partial(\bm{y}^{(i)})^{T}} 
    \frac{\partial \bm{y}^{(i)}}{\partial(\bm{\theta}^{(i)})^{T}}
\end{equation}
式子中$l$是损失函数,我们的目标是将损失函数最小化。计算过程被称为损失函数的反向传播。因为在这个过程中,损失函数从最后一层传播到第一层。
这种方法经常会遇到梯度消失问题,这意味着,梯度$ \frac{\partial l}{\partial(\bm{\theta}^{(i)})^{T}}$变得非常小,即当梯度从最后一层传播到第一层时, $\|\frac{\partial l}{\partial(\bm{\theta}^{(i)})^{T}}\|\rightarrow 0$(很快趋向0)。sigmoid激活函数在神经网络中是一种常用的激活函数,即每一层的函数$f^{(i)}$均对其输入进行逐元素的sigmoid激活。 \\
证明sigmoid激活函数很容易导致梯度消失问题。(提示:你可以观察在梯度下降过程中网络的某一个元素,也可以参考(2)画出的函数图像。)
\end{enumerate}
\end{problem}


%%%%%%%%%%%%%%%%%%%%%%%%%%%%%%%%%%%%%%%%%%%%%%%%%%%%%%%%%%%%%%%%%%%%%%%%%
% Problem 4
%%%%%%%%%%%%%%%%%%%%%%%%%%%%%%%%%%%%%%%%%%%%%%%%%%%%%%%%%%%%%%%%%%%%%%%%%%%%%%%%%%%%%%%%%%%%%%%%%%%%%%%%%%%%%%%%%%%%%%%%%%%%%%%%%%%%%%%%


\newpage
\noindent\rule{7in}{1pt}
\textbf{第二部分:实训题(共4分)}
\\ \\
\textbf{实训题要求:}
\begin{itemize}
    \item 本次作业包括1个实训题,作业要求以及基础代码以Aistudio项目的形式发布。
    \item 发布项目链接有效期3天,请在作业发布3天内fork这个项目,生成``我的项目'',并在自己fork的项目下进行作答,生成答案后按要求保存提交。
\end{itemize}

%%%%%%%%%%%%%%%%%%%%%%%%%%%%%%%%%%%%%%%%%%%%%%%%%%%%%%%%%%%%%%%%%%%%%%%%%
% Problem 1
%%%%%%%%%%%%%%%%%%%%%%%%%%%%%%%%%%%%%%%%%%%%%%%%%%%%%%%%%%%%%%%%%%%%%%%%%%%%%%%%%%%%%%%%%%%%%%%%%%%%%%%%%%%%%%%%%%%%%%%%%%%%%%%%%%%%%%%%

\begin{problem}{1 动物图片分类-构造神经网络}

在本作业中,首先给出了手写数字识别的示例。该示例中,展示了数据准备,网络配置和模型训练评估三部分的内容。阅读该示例可以帮助理解深度学习的一般过程。在动物图片分类的项目中,有以下任务需要完成,助教将根据任务完成的情况来给予分数。评判标准包括代码和实验结果两部分,实训题给分更关注代码逻辑和完整性。请在实验报告中写明实验结果和代码思路。

助教已经构建好了一个简单的卷积神经网络和训练流程,这是代码中已经实现的部分,你可以在这基础上进行改进。
以下是对于改进的一些提示:

\begin{itemize}
    \item 采用数据增强(data augmentation),对输入图片进行裁剪或者旋转等来提高性能。
    \item 改变网络结构,例如增加卷积层数,或者使用VGG等其他深度学习网络。注意:不要采用任何的预训练模型。(复杂的网络有过拟合风险,调整网络结构后超参需要随之调整。)
    \item 借助无标注数据,引入半监督学习(伪标签),观察无标注数据对分类性能的影响。
    \item 理解和补全ViT (Vision Transformer)模型,并分析实验结果。
\end{itemize}


实验介绍详情和参考基础代码请参见Aistudio中的共享项目\href{https://aistudio.baidu.com/studio/project/partial/verify/8989445/bf858e9af4984c35ba8157858e3dce89}{“人工智能作业二-动物图像分类”}。

\end{problem}

\noindent\rule{7in}{1pt}
\textbf{第三部分:实训题实验报告(共2分)}
\begin{itemize}
    \item 请按照实验报告模板完成实验报告。
    \item 实验报告模板是通用模板,可根据每个作业要求的差别,自由进行微调。
\end{itemize}

\end{CJK}
\end{document}